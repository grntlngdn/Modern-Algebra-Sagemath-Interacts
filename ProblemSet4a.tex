\documentclass[letterpaper,12pt]{article}
\usepackage{amsmath, amssymb}

\oddsidemargin=-.25in 
\evensidemargin=-.25in 
\topmargin=-.75in 
\textwidth=7in 
\textheight=10in

\def\p{\varphi}
\def\s{\sigma}
\def\R{\mathbb R}
\def\N{\mathbb N}
\def\Q{\mathbb Q}
\def\Z{\mathbb Z}
\def\C{\mathbb C}





\begin{document}
Let $D$ be the set of positive integers divisible by 4 (i.e $4\Z^+\{4,8,12,\dots\}$) together with regular addition and multiplication (note that both of these operations on elements of this set result in another element of this set).
Call $k$ an \emph{irreducible} of $D$ if it cannot be factored into a product of two elements of $D$, and \emph{reducible} otherwise (e.g. $2^5=(2^2)*(2^3)$ is reducible, but $3^22^3$ is irreducible in $D$). Note that for some $r$ reducible in $D$, $r$ has multiple factorizations into irreducibles (e.g. $2^3*2^3=2^2*2^2*2^2$).

\begin{enumerate}
\item Make a conjecture of the form "There will be precisely\emph{two} different ways to factor $r\in D$ as a product of irreducibles of $D$ if and only if \underline{\hspace{3cm}}," after observing factorization of various $r\in D$

\item  Prove a correct statement of this relationship, which will be given to you.

\end{enumerate}
(The correct statement is there will be precisely \emph{two} different ways to factor the integer $k\in D$ into irreducibles of $D$ if and only if one of three conditions hold: for $i=6,8,9,10,11,$ or $13$, $k=p^i$; for $q\ne p$ a prime and $i=5,6,$ or $7$, $k=qp^i$; or for primes $q_1\ne q_2\ne p$, $k=q_1q_2p^4$.)

\end{document}
